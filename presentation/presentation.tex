\documentclass{beamer}

\mode<presentation>
{
\usetheme{Dresden}

\setbeamercovered{transparent}
}

\usepackage[english]{babel}
\usepackage[latin1]{inputenc}
\usepackage{amsfonts}
\usepackage{amsmath}
\usepackage{mathtools}

\usepackage[scaled=.90]{helvet}
\usepackage{courier}
\usepackage{graphicx}
\usepackage{color}
\usepackage{subfig}
\DeclareGraphicsExtensions{.pdf,.png,.jpg,.mps}
\usepackage[absolute,overlay]{textpos}
\setlength{\TPHorizModule}{1mm}
\setlength{\TPVertModule}{1mm}
\usepackage{ragged2e}
\justifying
\usepackage{color}
\usepackage{physics}
\usepackage{mathtools}
\usepackage{amsmath}
\usepackage{bm}
\usepackage{calc}
\usepackage{listings}
\usepackage{url}
\usepackage{rotating}
%\usepackage{calrsfs}
\usepackage{amsfonts}
\usepackage[T1]{fontenc}

%%%% A NEW COMMAND TO FIX LOGO POSITION (x,y) in mm
\newcommand{\MyLogo}{%
\begin{textblock}{13}(88,74)
%  \pgfuseimage{logo}
 \includegraphics[height=1cm, angle=0]{images/pdp2017}
\end{textblock}
} 


%%%% A NEW COMMAND TO FIX LOGO POSITION (x,y) in mm

%%%%%%%%%%%%%%%%%%%%%%%%%%%%%%%%%%%%%%%%%%%%%%%%%%%%%%%%%%%%%%%%%%%%%%%%%

\title{A Tracking Algorithm for Particle-like Moving Objects}

\author{Davide Spataro, Paola Arcuri, Alessio De Rango, William Spataro, Donato D'Ambrosio\inst{1} and Alice Mari\inst{2}}
\institute[]{\inst{1} University of Calabria, Department of Mathematics and Computer Science \and %
\inst{2} Institute for BioEngineering University of Edinburgh}
\date{PDP 2017,  St. Petersburg, Russia\\
March 5-8, 2017}

\begin{document}

\begin{frame}
\MyLogo
\MyLogo
\titlepage
\end{frame}


\begin{frame}{Contents}
\tableofcontents
\end{frame}


\section{Introduction}
%Frame----------------------------------------------------------------------------------------------------------------------------------------------------
	\begin{frame}{Motivations}
				\begin{itemize}
					\item How bacteria react to a stimulus
					\item How a drug interfere with bacteria, changing their chemiotaxis i.e. change in movements
					\item For example monitoring how good is a drug impeding communication between cancer cells.
				\end{itemize}
				


	\end{frame}
	
	%Frame----------------------------------------------------------------------------------------------------------------------------------------------------
		\begin{frame}{Motivations}
				\begin{figure}
					\centering
					\includegraphics[scale=0.38]{./images/motivations.png}
				\end{figure}

				\begin{itemize}
					\item A: \textit{Normal} trajectories
					\item B: Drug is administred
					\item C: Red trajectories are of those who were affected by the drug
				\end{itemize}
				\textbf{Tracking allows  quantitatively analysis}!
	\end{frame}
	
%Frame----------------------------------------------------------------------------------------------------------------------------------------------------
		\begin{frame}{Motivations}
				\begin{figure}
					\centering
					\includegraphics[scale=0.30]{./images/motivations2.png}
				\end{figure}

				\begin{itemize}
					spiega che è sta cosa
				\end{itemize}

	\end{frame}
	
	
	%Frame----------------------------------------------------------------------------------------------------------------------------------------------------
		\begin{frame}{Motivations}
				\begin{figure}
					\centering
					\includegraphics[scale=0.30]{./images/motivations3.png}
				\end{figure}

				\begin{itemize}
				alga che muore e rilascia i nutrienti de quei cosi piccoli sono i batteri che si muovo per magnare
				\end{itemize}

	\end{frame}
	
	

\section{Tracking Algorithm}

	\subsection{Image Processing Using XCA}
%Frame----------------------------------------------------------------------------------------------------------------------------------------------------
		\begin{frame}{XCA - IFCA Engine }
			CA in generale			
		\end{frame}
		%Frame----------------------------------------------------------------------------------------------------------------------------------------------------
		\begin{frame}{XCA - IFCA Engine}
			\begin{itemize}
			\item Implemented on top of \textbf{OpenCAL}
			\item Each color channels is represented as a substate
			\item Local and Global Pixel transformations maps directly to \textit{elementary processes} and \textit{global operations}, respectively.
			\item Runs on a number of parallel architectures and accelerators
			\item Easy to use. Threshold Filter in ~$20$ LoC.
			\item Seamless input/output
			\end{itemize}
		\end{frame}
		%Frame----------------------------------------------------------------------------------------------------------------------------------------------------
		\begin{frame}{XCA - IFCA Engine}
			
		\end{frame}		
		
	%Frame----------------------------------------------------------------------------------------------------------------------------------------------------	
		\begin{frame}{IFCA Engine - Convolutional Operators}
			
		\end{frame}
%Frame----------------------------------------------------------------------------------------------------------------------------------------------------
		\begin{frame}{IFCA Engine - Example: $N-$Dimensional Laplacian Filter}
			
		\end{frame}	
		
	\subsection{Trajectories Reconstruction}	
%Frame----------------------------------------------------------------------------------------------------------------------------------------------------	
		\begin{frame}{Objective}
			\begin{block}{Algorithm}
				From a set of particle representations construct a set of trajectories over time.
			\end{block}
			
		\end{frame}
	%Frame----------------------------------------------------------------------------------------------------------------------------------------------------	
	


\section{Test Case: Motility of B. Subtilis}
%Frame----------------------------------------------------------------------------------------------------------------------------------------------------
	\begin{frame}{Experiment Setting}

	\end{frame}
%Frame----------------------------------------------------------------------------------------------------------------------------------------------------
	\begin{frame}{How Bacteria Moves?}

	\end{frame}
%Frame----------------------------------------------------------------------------------------------------------------------------------------------------




\begin{frame}%%     2
\begin{center}
{\fontsize{40}{50}\selectfont Thank You!}
\end{center}
\end{frame}

\end{document}