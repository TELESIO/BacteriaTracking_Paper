\subsection{Cellular Automata}\label{sec:CA}
    Cellular Automata are parallel computing models, whose evolution
    is defined by local rules. A cellular automaton can be thought as
    a $d$-dimensional space, called \emph{cellular space}, subdivided
    in regular cells of uniform shape and size. Each cell embeds a
    \emph{finite automaton}, one of the most simple and well known
    computational models, which can assume a finite number of
    states. At time $t=0$, cells are in arbitrary states and the CA
    evolves step by step by changing the states of the cells at
    discrete time steps, by applying the same local rule of evolution,
    i.e. the cell's \emph{transition function}, simultaneously
    (i.e. in parallel) to each cell of the CA. Input for the cell is
    given by the states of a predefined (usually small) set of
    neighboring cells, which is assumed invariant in space and time.
 
 Extended Cellular Automata\cite{DiGregorio&Serra-1999}  represents an extension of the original CA computational paradighm.   
    The main differences between XCA and classical CA are in that the state can be expressed as Cartesian product of the $n$ 
    \emph{substates}, the transition function can also be decomposed into \textit{elementary processes} that can be parametrized,
    and non-local operation, that go under the name of \textit{global functions} are allowed.
        The initial conditions of the system are obtained by preliminarly
    applying a non-local initialization operation.
    

    The framework is defined as a 7-tuple $ A = <R,X,Q,P,\sigma,\Gamma,\gamma>$ where:    
    $R$ is a finite discrete $2$dimensional space, $\Gamma=R$ is the region over the global operation are applied,
    $X=X(x_0,y_0)=\{(x,y): \left| x-x_0\right| \leq r \wedge \left| y-y_0\right| \leq r\} $ defines the \textit{Moore}'s neighborhood relationship of radius $r$, $P=\emptyset$, $\psi$ is the initialization function, $\sigma=\{\sigma_i:Q^{|X|} \mapsto Q\}$ and   $\gamma= \{\gamma_i:Q^{|\Gamma|} \rightarrow Q^{|\Gamma|}\} $ are the set of  elementary processes non-local functions, respectively.
    

%    \begin{itemize}
%
%    \item $R$ is the $d$-dimensional cellular space.
%
%    \item $\Gamma \subseteq R$ is the region over which steering\footnote{Global operations on the entire cellular space (e.g. to model external
%      influences that can not easily be described in terms of local
%      interactions, or to perform reductions over the whole, or a subset
%      of, the cellular space).}  is applied.
%
%    \item $X$ is the geometrical pattern that specifies the neighborhood
%      relationship; $m = |X|$ represent the number of elements in the set
%      $X$, i.e. the number of neighbors for the central cell.
%
%    \item $Q = Q_0 \times Q_1 \times....\times Q_{n-1}$ is the set of
%      cell's states, expressed as Cartesian product of the $n$ considered
%      \emph{substates} $Q_0 \times Q_1 \times....\times Q_{n-1}$.
%
%    \item $P = {p_0,p_1,....,p_{p-1}}$ is the set of CA
%      \emph{parameters}.They can allow a fine tuning of the XCA model,
%      with the purpose of reproducing different dynamical behaviors of
%      the phenomenon of interest.
%
%    \item $\psi : Q^{|R|} \rightarrow Q^{|R|}$ is the global function
%      which define the initial conditions of the system.
%      
%    \item $\sigma : Q^m \rightarrow Q$ is the cell's transition function.
%      It is split in $s$ \emph{elementary processes}, $\sigma_0,\sigma_1,
%      ..., \sigma_{s-1}$, each one describing a particular aspect ruling
%      the dynamic of the considered system.
%
%    \item $\gamma: Q^{|\Gamma|} \rightarrow Q^{|\Gamma|} \times
%      \mathbb{R}$ is the (global) steering function.
%
%    \end{itemize}


