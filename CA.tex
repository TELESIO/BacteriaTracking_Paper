\subsection{Cellular Automata}\label{sec:CA}
    Cellular Automata are parallel computing models, whose evolution
    is defined by local rules. A cellular automaton can be thought as
    a $d$-dimensional space, called \emph{cellular space}, subdivided
    in regular cells of uniform shape and size. Each cell embeds a
    \emph{finite automaton}, one of the most simple and well known
    computational models, which can assume a finite number of
    states. At time $t=0$, cells are in arbitrary states and the CA
    evolves step by step by changing the states of the cells at
    discrete time steps, by applying the same local rule of evolution,
    i.e. the cell's \emph{transition function}, simultaneously
    (i.e. in parallel) to each cell of the CA. Input for the cell is
    given by the states of a predefined (usually small) set of
    neighboring cells, which is assumed invariant in space and time.
 
 Extended Cellular Automata \cite{DiGregorio&Serra-1999} represents
    an extension of the original CA computational paradighm. 



   

    A XCA is defined as a 7-tuple $ A = <R,X,Q,P,\sigma,\Gamma,\gamma>$ where:

    \begin{itemize}

    \item $R$ is the $d$-dimensional cellular space.

    \item $\Gamma \subseteq R$ is the region over which steering\footnote{Global operations on the entire cellular space (e.g. to model external
      influences that can not easily be described in terms of local
      interactions, or to perform reductions over the whole, or a subset
      of, the cellular space).}  is applied.

    \item $X$ is the geometrical pattern that specifies the neighborhood
      relationship; $m = |X|$ represent the number of elements in the set
      $X$, i.e. the number of neighbors for the central cell.

    \item $Q = Q_0 \times Q_1 \times....\times Q_{n-1}$ is the set of
      cell's states, expressed as Cartesian product of the $n$ considered
      \emph{substates} $Q_0 \times Q_1 \times....\times Q_{n-1}$.

    \item $P = {p_0,p_1,....,p_{p-1}}$ is the set of CA
      \emph{parameters}.They can allow a fine tuning of the XCA model,
      with the purpose of reproducing different dynamical behaviors of
      the phenomenon of interest.

    \item $\psi : Q^{|R|} \rightarrow Q^{|R|}$ is the global function
      which define the initial conditions of the system.
      
    \item $\sigma : Q^m \rightarrow Q$ is the cell's transition function.
      It is split in $s$ \emph{elementary processes}, $\sigma_0,\sigma_1,
      ..., \sigma_{s-1}$, each one describing a particular aspect ruling
      the dynamic of the considered system.

    \item $\gamma: Q^{|\Gamma|} \rightarrow Q^{|\Gamma|} \times
      \mathbb{R}$ is the (global) steering function.

    \end{itemize}

    The initial conditions of the system are obtained by preliminarly
    applying the $\psi$ initialization function. Then, if $\tau$
    represents the function which applyes the transition function
    $\sigma$ to all the cells in $R$, as for the CA case, and $\gamma$
    the XCA steering function, the XCA evolution is obatained by
    applying the $\Phi = \tau \circ \gamma$ XCA global transition
    function at discrete time steps.
